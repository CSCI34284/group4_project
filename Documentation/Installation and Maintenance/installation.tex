\documentclass[11pt]{article}

\usepackage{color}
\usepackage{xcolor}
\usepackage{enumitem}
\usepackage{fancyhdr}
\usepackage{fancyvrb}
\usepackage{listings}
\usepackage{amsmath}
\usepackage{float}
\usepackage[T1]{fontenc}
\usepackage[utf8]{inputenc}
\usepackage{pmboxdraw}
\usepackage{graphicx}
\usepackage{newunicodechar}
\newunicodechar{└}{\textSFii}
\newunicodechar{├}{\textSFviii}
\newunicodechar{─}{\textSFx}
\newunicodechar{│}{\textSFxi}
\usepackage[utf8]{inputenc}
\usepackage[obeyspaces]{url}
\usepackage[margin=1.25in]{geometry}
\usepackage[russian,english]{babel}

\usepackage{tcolorbox}
\renewcommand{\labelitemii}{$\circ$}

\pagestyle{fancy}
\lfoot{\small\scshape Installation}
\cfoot{}
\rfoot{\footnotesize \thepage}
\lhead{\small\scshape CSCI 3428}
\chead{}
\rhead{\footnotesize Software Engineering}
\renewcommand{\headrulewidth}{.3pt}
\renewcommand{\footrulewidth}{.3pt}
\setlength\voffset{-0.25in}
\setlength\textheight{648pt}

\definecolor{dkgreen}{RGB}{45,127,44}
\definecolor{dkblue}{RGB}{39,42,126}
\definecolor{black}{RGB}{0,0,0}

\lstset{
  frame=single,
  framesep=10pt,
  aboveskip=3mm,
  belowskip=3mm,
  showstringspaces=false,
  columns=flexible,
  basicstyle={\small\ttfamily\color{dkblue}},
  rulecolor=\color{black},
  breaklines=true,
  breakatwhitespace=true,
  tabsize=3,
  moredelim=**[is][\color{dkgreen}]{@}{@},
  moredelim=**[is][\color{maroon}]{!}{!}
}

\newsavebox{\FVerbBox}
\newenvironment{FVerbatim}
 {\VerbatimEnvironment
  \begin{center}
  \begin{BVerbatim}}
 {\end{BVerbatim}
  \end{center}}

\begin{document}

\title{CSCI 3428 - Installation and Maintenance Guide}
\author{Group 4}
\date{Wednesday 11\textsuperscript{th} December, 2019}
\maketitle

% Hides header but displays footer on first page
\fancypagestyle{plain}{
\fancyhf{} % clear all header and footer fields
\fancyfoot[r]{\footnotesize \thepage} % except the right
\fancyfoot[l]{\small\scshape Installation} % and left
\renewcommand{\headrulewidth}{0pt}
\renewcommand{\footrulewidth}{.3pt}}

%\def\UrlFont{\bfseries\rmfamily} % makes URLs bold

\section{User Access}
The project will be hosted on the undergraduate computer science student's server at Saint Mary's
University, and is publicly accessible at: \url{ugdev.cs.smu.ca/~group4}. The project has been
developed and tested on the latest desktop versions of Chrome (\url{v.78}), Firefox (\url{v.70}),
and Safari (\url{v.13.0.1}), though should also be compatible with older versions and mobile
platforms, provided there is support for HTML5, CSS3, and Javascript.

On access, existing users will be prompted to log into the messaging system with their given
account. The information entered into the appropriate text-fields will be validated against the
credentials stored in the system's database, and allow or reject access to the system as
appropriate. By default, the user's credentials are the following:
\begin{FVerbatim}
Username: FIRST_NAME
Password: ──────────
\end{FVerbatim}
Authenticating new users, changing passwords, and adjusting any other user-related settings
will be done by the administrator using the user-management panel that is run on the server, and is
accessible via the web.

\section{Administration Panel}

Most day-to-day management of the system will be performed by the administrator using the
included user-management panel. This includes adding new users or removing existing ones from the
system, and making connections between users in the form of new conversations. Currently, the
administration panel is running on port 8000 of the undergraduate computer science student's server
at Saint Mary's university. It can be accessed by navigating to: \url{ugdev.cs.smu.ca:8000/admin/}.
\newline

\begin{tcolorbox}
\textbf{IMPORTANT:} via the administration panel, the entire conversation history of all users of
the system is visible. There further exists functionality to change or delete the contents of
previously sent messages. Malicious use of this panel threatens the privacy of the system's users,
and must be protected against accordingly. The default password given above must be changed
to a more secure alternative, and access to the administration panel must be limited exclusively to
the system administrator.
\end{tcolorbox}

\pagebreak
\noindent To change the host port for the panel, begin by killing any active tasks on the desired
port (e.g. 4433):
\begin{center} \begin{tabular}{c} \begin{lstlisting}[linewidth=3.7cm]
@$@ fuser -k 4433/tcp
\end{lstlisting} \end{tabular} \end{center}
Then, from within \url{~/project_master/backend/smumessaging}, initiate the panel by running a new task
on the desired port (e.g. 9000):
\begin{center} \begin{tabular}{c} \begin{lstlisting}[linewidth=11.6cm]
@$@ nohup ./redis-server --protected-mode no &
@$@ daphne -b 192.168.2.12 -p 9000 smumessaging.asgi:application
\end{lstlisting} \end{tabular} \end{center}
The panel can then be accessed by navigating to \url{ugdev.cs.smu.ca:9000/admin/}, using the
following log-in credentials:
\begin{FVerbatim}
Username: ──────────
Password: ──────────
\end{FVerbatim}

\subsubsection*{User}
From within the administration panel, add a new user by navigating to the \url{Users} tab, and selecting
\url{Add user}. Enter the desired log-in credentials, and click \url{Save} to confirm. The table of
existing users is listed at the bottom of the page. Note that \url{Staff Status=True} indicates
that the user is able to access the administration panel using their login credentials. As per
above, the sole staff member should be the system administrator.

To delete a user, select the tick box next to their name in the table at the bottom of the
\url{Users} page. Change the action drop-down to \url{Delete selected users}, and press \url{Go} to
confirm.

To change a user's password and permissions, select the name of the user from the table. Fill out
the form provided in the first box to change their password, and use the scroll-menu in the
\url{Permissions} section to alter their permissions. Information concerning the date the user was
created, and the last time they accessed the system is available at the bottom of this page.

\subsubsection*{Chats}
From the administration panel, initiate a new conversation by navigating to the \url{Chats} tab.
Select \url{Add chat}, and select the two (or more) desired participants from within the
\url{Participants} scroll-menu. Click the \url{Save} button at the bottom of the page to confirm.
Note that it is possible to populate this conversation with existing messages via the \url{Messages}
menu. However, this is an artifact of the design of the administration pane, and is not intended to
(and indeed should not) be used.

All active conversations are enumerated in the table at the bottom of the \url{Chats} page, which
can further be used to delete a conversation.

\subsubsection*{Other}
The design of the administration panel is such that it allows additional functionality, including
changing and deleting messages, creating groups, changing a user's authorisation token, etc. Many of
these features are not implemented in the system itself, and can therefore be ignored.

\section{Server Access}
The project's files, databases, and server configuration can be accessed via \url{SSH}:
\begin{FVerbatim}
Username: ──────────
Password: ──────────
\end{FVerbatim}
Currently, all the tools and databases required for managing the project are installed on the
server. This includes \url{v.3.6.8} of Python3, which is used to install and manage the Django
framework, and the latest version of SQLite. With SQLite's integration in the Django user-management
platform, direct access to the project's databases is not required, and any necessary modifications
can again be performed by the system-administrator in the administration panel, as outlined above.

The ReactJS framework was used to create the source files for the project's front-end, and while it
is not required for the basic day-to-day running and administration of the system, any long-term
changes to the project (including bug-fixes and adding functionality) will require that the server
has a suitable JavaScript development environment available. See the installation procedure below
for setting up a new environment.

\subsection{Project Directories}
The high-level directory structure for the server is outlined below. The web-facing \url{public_html}
directory contains an optimised build version of the project, that condenses the various source and
package files. While fully functional, these files are not (easily) readable, and are not intended
to be modified directly. Changes should first be made to the source-code within the project
directory, tested appropriately, and lastly built and included in \url{public_html}.
\begin{FVerbatim}[fontsize=\small]
~
├── project_master
│   ├── backend
│   └── messagingSystem
└── public_html
    ├── index.css
    ├── index.html
    └── index.js
\end{FVerbatim}
The system files are split into the front- and back-end components within the \url{project_master}
directory. The \url{backend} directory is responsible for initialising the database and populating
its tables with user input, handling user communication, and specifying any dependencies for the
Django project. Some files and directories of note:
\begin{itemize}
    \item \url{/chat} Configures the system's server
    \begin{itemize}
        \item \url{/admin.py} Initialises tables for display in administration panel
        \item \url{/api/serializers.py} Formats data for entry into database tables
        \item \url{/consumers.py} Handles message sync. on load + sending/receiving messages
        \item \url{/models.py} Database table initialisation and structure
        \item \url{/routing.py} Defines URL and connection to websocket
    \end{itemize}
    \item \url{/manage.py} Initialises Django's CLI (auto-generated)
    \item \url{/smumessaging} Django configuration (auto-generated)
\end{itemize}

\begin{FVerbatim}[fontsize=\small]
project_master
└── backend
    ├── chat
    │   ├── __init__.py
    │   ├── __pycache__
    │   ├── admin.py
    │   ├── api
    │   │   ├── __init__.py
    │   │   ├── __pycache__
    │   │   ├── serializers.py
    │   │   ├── urls.py
    │   │   └── views.py
    │   ├── apps.py
    │   ├── consumers.py
    │   ├── migrations
    │   ├── models.py
    │   ├── routing.py
    │   ├── tests.py
    │   ├── urls.py
    │   └── views.py
    ├── db.sqlite3
    ├── manage.py
    ├── media
    ├── package-lock.json
    ├── package.json
    ├── requirements.txt
    ├── runtime.txt
    └── smumessaging
        ├── __init__.py
        ├── __pycache__
        ├── asgi.py
        ├── routing.py
        ├── settings.py
        ├── urls.py
        └── wsgi.py
\end{FVerbatim}
The \url{messagingSystem} directory contains the source-files for the system's front-end, which
specifies the structure and appearance of the web-page, and reads input from the users. Some files
and directories of note include:
\begin{itemize}
    \item \url{/public} Base files for system structure and appearance (linked to by other files)
    \item \url{/src} Main system directory
        \begin{itemize}
            \item \url{/components} Contains CSS and ReactJS files for general + user-specific pages
                \begin{itemize}
                    \item \url{/Communication.css} Specifies appearance of conversation panel
                    \item \url{/Communication.js} Specifies structure of conversation panel
                    \item \url{/Login.css} Specifies appearance of login page
                    \item \url{/Login.js} Specifies structure of login page
                    \item \url{/UserInterface.css} Base appearance for interface
                    \item \url{/UserInterface.js} Base structure for interface
                \end{itemize}
            \item \url{/index.js} Sub-initialisation
            \item \url{/models} Handles communication with backend and websocket integration.
            \item \url{/router.js} Handles navigation between pages
            \item \url{/routes/IndexPage.js} Handles user login and authentication
            \item \url{/services/global.js} Handles requests from server
            \item \url{utils/request.js} Handles return data from server
        \end{itemize}
\end{itemize}

\begin{FVerbatim}[fontsize=\small]
project_master
└── messagingSystem
    ├── package-lock.json
    ├── package.json
    ├── public
    │   ├── index.css
    │   └── index.html
    └── src
        ├── assets
        │   └── bg2.jpg
        ├── components
        │   ├── BobCommunication.css
        │   ├── BobUserInterface.css
        │   ├── Communication.css
        │   ├── Communication.js
        │   ├── Login.css
        │   ├── Login.j
        │   ├── MayCommunication.css
        │   ├── MayUserInterface.css
        │   ├── UserInterface.css
        │   ├── UserInterface.js
        │   ├── ValerieCommunication.css
        │   └── ValerieUserInterface.css
        ├── index.css
        ├── index.js
        ├── models
        │   ├── IndexPage.js
        │   └── UserInterface.js
        ├── router.js
        ├── routes
        │   └── IndexPage.js
        ├── services
        │   └── global.js
        ├── settings.js
        ├── utils
        │   └── request.js
        └── websocket.js
\end{FVerbatim}

\section{Migration and Maintenance}
\subsection{Django and Server}
In order to migrate the project to a new server (or make and push any changes to the source-code),
the host machine will require that Python3, MySQL, and a JavaScript development environment is
installed. Ensure first that the Client-URL (\url{curl}) command-line utility for Linux is
installed:
\begin{center} \begin{tabular}{c} \begin{lstlisting}[linewidth=1.75cm]
@$@ curl -V
\end{lstlisting} \end{tabular} \end{center}
If not, install it using your system's appropriate package-installation tool. For Debian, Ubuntu,
or related distributions, use:
\begin{center} \begin{tabular}{c} \begin{lstlisting}[linewidth=5cm]
@$@ sudo apt-get install curl
\end{lstlisting} \end{tabular} \end{center}
Most Linux distributions will include Python3 by default. Verify that it is installed, and install
it if not. Preferably, select a distribution that includes the Python package manager pip, which
will be used to install the Django framework. Otherwise, it can be obtained separately:
\begin{center} \begin{tabular}{c} \begin{lstlisting}[linewidth=10.8cm]
@$@ curl https://bootstrap.pypa.io/get-pip.py -o get-pip.py
@$@ python get-pip.py
@$@ pip install Django
\end{lstlisting} \end{tabular} \end{center}
To configure the project's database, ensure that SQLite (\url{v.3.30.1} or later) is installed on the
host-machine. The system's tables are structured as follows:
\begin{figure}[H]
\includegraphics[scale=0.65]{tables.png}
\centering
\end{figure}
To run the server on the new host-machine, begin by ensuring that all required dependencies are
first installed:
\begin{center} \begin{tabular}{c} \begin{lstlisting}[linewidth=6cm]
@$@ pip3 install -r requirements.txt
@$@ pip3 install django-allauth
@$@ pip3 install channels
@$@ pip3 install django-rest-auth
@$@ pip3 install channels_redis
@$@ pip3 install "whitenoise <4"
@$@ pip3 install service_identity
\end{lstlisting} \end{tabular} \end{center}
Lastly, run the server and backend:
\begin{center} \begin{tabular}{c} \begin{lstlisting}[linewidth=5.2cm]
@$@ redis-server
@$@ python manage.py runserver
\end{lstlisting} \end{tabular} \end{center}

\subsection{React and Front-end}
\noindent In order to maintain the project's source code, a JavaScript development environment will
need to be present on the host machine (in particular, the React framework). Ensure that the
newest version of the Node.js runtime environment is installed. If not, install Node.js and additional
build tools:
\begin{center} \begin{tabular}{c} \begin{lstlisting}[linewidth=7.5cm]
@$@ sudo apt-get install nodejs
@$@ sudo apt-get install build-essentials
\end{lstlisting} \end{tabular} \end{center}
As of \url{v.8.10} of Node, and \url{v.5.6} of NPM (bundled in the Node installation), a new React
project can be created using:
\begin{center} \begin{tabular}{c} \begin{lstlisting}[linewidth=5.5cm]
@$@ npx create-react-app my-app
@$@ cd my-app
@$@ npm start
\end{lstlisting} \end{tabular} \end{center}
A local instance of the project's front-end structure can be run locally to test changes to the
structure or the appearance from within the \url{messagingSystem} directory. Use NPM to first
install any missing dependencies, and then start the application:
\begin{center} \begin{tabular}{c} \begin{lstlisting}[linewidth=2.5cm]
@$@ npm install
@$@ npm start
\end{lstlisting} \end{tabular} \end{center}

\end{document}
