\documentclass[11pt]{article}

\usepackage{color}
\usepackage{xcolor}
\usepackage{enumitem}
\usepackage{scrextend}
\usepackage{fancyhdr}
\usepackage[utf8]{inputenc}
\usepackage[obeyspaces]{url}
\usepackage[margin=1.25in]{geometry}

\pagestyle{fancy}
\lfoot{\small\scshape SDP}
\cfoot{}
\rfoot{\footnotesize \thepage}
\lhead{\small\scshape CSCI 3428}
\chead{}
\rhead{\footnotesize Software Engineering}
\renewcommand{\headrulewidth}{.3pt}
\renewcommand{\footrulewidth}{.3pt}
\setlength\voffset{-0.25in}
\setlength\textheight{648pt}

\begin{document}

\title{CSCI 3428 - Software Development Plan}
\author{Group 4}
\date{Monday 7 \textsuperscript{th} October 2019}
\maketitle

\fancypagestyle{plain}{
\fancyhf{} % clear all header and footer fields
\fancyfoot[r]{\footnotesize \thepage} % except the right
\fancyfoot[l]{\small\scshape SDP} % and left
\renewcommand{\headrulewidth}{0pt}
\renewcommand{\footrulewidth}{.3pt}}

\section{Schedule}

See figure on final page.

\section{Team Organisation}
With a larger than standard group, we are able to delegate more members to work on the front-end of
the project, focusing specifically on creating a user-interface that is well-suited to the needs of
each of the individual clients. The underlying core functionality will likely be the same for each
of the users, and so fewer back-end developers are required. A larger group also grants more
flexibility for group-members to move back and forth between the different development areas, as
there is unlikely to be a shortage of people able to work on any one aspect of the project at any
given time. This means that, if a member decides they would like to move from front-end to back-end
to implement a novel idea, they are free to do so (within reason).

A larger group presents management challenges, however, and can slow the ideation and implementation
processes. As a result, we are tentatively considering dividing into two smaller sub-teams, that
will each focus independently on development for a specific client. This will likely
happen later in the development process, once a baseline for the UI and functionality has been
established, and the focus turns to tuning the software according to the client’s needs.

\subsection{Roles}
\textbf{Leader(s):}
\begin{addmargin}[1em]{2em} Undecided \end{addmargin}\vspace{2mm}
\textbf{Front-end Developers:}
\begin{addmargin}[1em]{2em} Gautham Chalapathy, Lulu Chen, Akhtar Rafid, Yilin Zhang
\end{addmargin}\vspace{2mm}
\textbf{Back-end Developers:}
\begin{addmargin}[1em]{2em} Veronica Hatala, Martin McLaren, Sheldon Taylor, Kreetish Venkataswami
\end{addmargin}\vspace{2mm}
\textbf{Server-side Developers:}
\begin{addmargin}[1em]{2em} Sarah Clarke, Martin McLaren, Kreetish Venkataswami
\end{addmargin}\vspace{2mm}
\textbf{Specialists:}
\begin{addmargin}[1em]{2em} Gautham Chalapathy, Sarah Clarke \end{addmargin}\vspace{2mm}

\section{Configuration Management Plan}

\subsection{GitHub}
Given the scale of the project and the size of the group, a mechanism for sharing code is required
to ensure that development progresses efficiently. GitHub is an obvious choice for such a platform,
and will act as the main hub for the project. Other attendant benefits include the ability to track
any changes made to the code, revert to an earlier version in case of serious error, and ensure that
all progress is saved in the event of data loss. Lastly, the ubiquity of the platform and the number
of resources available that document its functionality means the overhead cost of learning to use
GitHub is relatively small compared to the benefits gained.

\subsection{Agile}
From a group-management perspective, we plan to broadly follow the Agile framework. Outside of
providing a general strategy for managing the development process, an iterative approach is
particularly well suited for the schedule we are following. Any existing functionality (as will be
given in the early prototypes, for example) will need to be changed as both the clients
understanding of the product and our understanding of their specific needs develop.

\subsection{Django}
Lastly, we are considering building our web application using the Django framework. In general,
developing under a web framework ensures a consistency in both the design and functionality of the
various aspects of our project (through the use of standard libraries, for instance). They
additionally provide tools that makes creating databases and handling user management more
efficient, both of which are key challenges our project faces. Django in particular is well-suited
for our needs, as it uses a Python back-end that is relatively simple to learn and automatically
solves many of the various security considerations.

\section{Risk Management Plan}

While it is difficult to assess potential risks while the details of the group's organisation and
ultimate goal are still being assessed, we can identify several general obstacles that
we might face:

\subsection{Withdrawal}
\textit{Probability:} Low (\textless20\%) \begin{addmargin}[1em]{2em} Given the nature of the course
(third year, no examinations, single large-scale project), and that we are a third of the way
through the semester, it is unlikely that a group member will withdraw from the course going
forward.\end{addmargin}\vspace{2mm}
\noindent \textit{Impact:} Low (2/5) \begin{addmargin}[1em]{2em}Though a withdrawal would
typically have a large negative impact on the progress of the group’s development, a large group
means that a single member’s responsibilities can be distributed across the team without putting too
much burden on the other members.\end{addmargin}\vspace{2mm}
\noindent \textit{Mitigation:} \begin{addmargin}[1em]{2em} Discuss with the team prior to delegating
responsibilities and beginning development to assess any potential challenges they might face both
in the course and outside of it that might push them to consider withdrawing, and
discuss possible solutions (assigning less work, getting extra support, etc.).\end{addmargin}

\subsection{Data Loss}
\textit{Probability:} Medium (50\%) \begin{addmargin}[1em]{2em} The loss of data, whether through
broken, lost, or faulty equipment, is a common occurrence, and one that is difficult to predict.
\end{addmargin}\vspace{2mm}
\noindent \textit{Impact:} High (4/5) \begin{addmargin}[1em]{2em} Without taking the proper
precautions, data loss presents one of the most substantial risks to the project, threatening
weeks of progress, and resulting in large set backs, or an inability to meet the requirements.
\end{addmargin}\vspace{2mm}
\noindent \textit{Mitigation:} \begin{addmargin}[1em]{2em} Protecting against data loss is becoming
increasingly easier with the wide availability of cloud services and the reduced cost of storage
media. Additionally, distributing new code to each member of the group using GitHub lessens any
potential impact.\end{addmargin}

\subsection{Late-Stage Changes in Client's Needs}
\textit{Probability:} High (\textgreater60\%) \begin{addmargin}[1em]{2em} A client’s needs change on
a regular basis, and this is particularly true for a product that is custom-designed for them. While
such a change is not itself a problem, a large-scale change late in the development cycle can pose a
significant setback.\end{addmargin}\vspace{2mm}
\noindent \textit{Impact:} Medium (3/5) \begin{addmargin}[1em]{2em} Depending on the
scale, and the time during the development cycle in which they are brought up,
this could represent a serious setback to the group, potentially requiring ground-up changes in
functionality or design. Alternatively, small-scale changes can be implemented at virtually any
point in the development cycle at little extra cost.\end{addmargin}\vspace{2mm}
\noindent \textit{Mitigation:} \begin{addmargin}[1em]{2em} Involve clients at every point of the
development cycle (within reason), determine a set of core needs that won’t change, and work on
addressing these first. Communicate with clients what a reasonable set of expectations for the
product looks like, and what the team is able to achieve given the time constraints. \end{addmargin}

\subsection{Failure to Meet Deadline}
\textit{Probability:} Very Low (\textless10\%) \begin{addmargin}[1em]{2em} With the size of the
group, the importance of the project, and the schedule created, it is unlikely that the base project
will be incomplete by the deadline.\end{addmargin}\vspace{2mm}
\noindent \textit{Impact:} Very High (5/5) \begin{addmargin}[1em]{2em} A non-functioning product, or
one that doesn't meet the client's specifications represents a significant failure to complete the
project. \end{addmargin}\vspace{2mm}
\noindent \textit{Mitigation:} \begin{addmargin}[1em]{2em} Create a detailed plan that specifies the
key features of the project. Ensure each group member has a clear idea of what their
responsibilities are, and where in the development process they should be at various stages.
Ensure resources are allocated appropriately, and address any problems as they
arise.\end{addmargin}

\section{Installation and Maintenance Plan}

The project is a web application that will be hosted on the undergraduate computer science student's
server at Saint Mary's University (\url{ugdev.cs.smu.ca}). The details of how the project's
databases and server-side components will be implemented will be decided at a later stage.
In addition, documentation will be created that details the purpose of each component of the
project, including both its structure and its functionality. This documentation will be available
within the source code, and also as a separate document that can be found on the project's GitHub
repository.

\end{document}
