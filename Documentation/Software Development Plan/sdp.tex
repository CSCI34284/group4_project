\documentclass[11pt]{article}

\usepackage{color}
\usepackage{xcolor}
\usepackage{enumitem}
\usepackage{scrextend}
\usepackage{graphicx}
\usepackage{float}
\usepackage{fancyhdr}
\usepackage{fancyvrb}
\usepackage{listings}
\usepackage[utf8]{inputenc}
\usepackage[obeyspaces]{url}
\usepackage[margin=1.25in]{geometry}

\pagestyle{fancy}
\lfoot{\small\scshape SDP}
\cfoot{}
\rfoot{\footnotesize \thepage}
\lhead{\small\scshape CSCI 3428}
\chead{}
\rhead{\footnotesize Software Engineering}
\renewcommand{\headrulewidth}{.3pt}
\renewcommand{\footrulewidth}{.3pt}
\setlength\voffset{-0.25in}
\setlength\textheight{648pt}

\begin{document}

\title{CSCI 3428 - Software Development Plan}
\author{Group 4}
\date{Monday 18\textsuperscript{th} November 2019}
\maketitle

\fancypagestyle{plain}{
\fancyhf{} % clear all header and footer fields
\fancyfoot[r]{\footnotesize \thepage} % except the right
\fancyfoot[l]{\small\scshape SDP} % and left
\renewcommand{\headrulewidth}{0pt}
\renewcommand{\footrulewidth}{.3pt}}

\section{Schedule}

See figures on final page.

\section{Team Organisation}
With a larger group, it is possible to delegate more members to work on the front-end of the project,
focusing specifically on creating a user-interface that is well-suited to the needs of each of the
individual clients. The underlying core functionality will likely be the same for each of the users,
and so fewer back-end developers are required. A larger group also grants more flexibility for
group-members to move back and forth between the different development areas, as there is unlikely
to be a shortage of people able to work on any one aspect of the project at any given time. This
means that, if a member decides they would like to move from front-end to back-end to implement a
novel idea, they are free to do so (within reason).

A larger group presents management challenges, however, and can slow the ideation and implementation
processes. As a result, one possible consideration involves dividing the group into two smaller
sub-teams, that will each focus independently on development for a specific client. This will likely
happen later in the development process, once a baseline for the UI and functionality has been
established, and the focus turns to tuning the software according to the client’s needs.

\subsection{Roles}
\textbf{Leader(s):}
\begin{addmargin}[1em]{2em} Veronica Hatala \end{addmargin}\vspace{2mm}
\textbf{Front-end Developers:}
\begin{addmargin}[1em]{2em} Gautham Chalapathy, Lulu Chen, Akhtar Rafid, Yilin Zhang
\end{addmargin}\vspace{2mm}
\textbf{Back-end Developers:}
\begin{addmargin}[1em]{2em} Veronica Hatala, Martin McLaren, Brian Philip, Sheldon Taylor, Kreetish
Venkataswami
\end{addmargin}\vspace{2mm}
\textbf{Server-side Developers:}
\begin{addmargin}[1em]{2em} Sarah Clarke, Martin McLaren, Brian Philip, Sheldon Taylor, Kreetish
Venkataswami
\end{addmargin}\vspace{2mm}
\textbf{Specialists:}
\begin{addmargin}[1em]{2em} Gautham Chalapathy, Sarah Clarke \end{addmargin}\vspace{2mm}

\section{Proposed Standards, Procedures, Techniques, and Tools}
\subsection{Waterfall w/ Phased Development}
From a group-management perspective, development of the project will occur under the Waterfall
framework. While an iterative approach provides a standard for adding functionality as the clients'
understanding of the product develops, phased development streamlines the management of the
development process into a linear sequence of steps, in turn reducing the complexity of the project.
Additionally, rigorous planning mitigates the impact of difficulties faced in the program's
development. Lastly, the Waterfall model emphasises the importance of creating detailed
documentation outlining the system's functionality, design considerations, and long-term management.
Given that maintenance of the project upon its completion will not fall to the team's members means
having such documentation will be especially valuable.

\subsection{React}
The front-end of the project will be built up using the React library for JavaScript. React
allows us to encapsulate the layout and structure of the program within a JavaScript function (or
functions). Doing so allows the front-end (and in particular, input from the users) to be posted
directly to the server, and in general does away with any intermediary steps in the communication
between the program on the user's end, and the server and database. This is particularly helpful in
instant messaging applications, which require frequent communication with the hosting
server/database.

\subsection{Django}
User-management, security, and communication between the front-end and the server will be done using
the Django framework. In general, developing under a web framework ensures a consistency in both the
design and functionality of the various aspects of our project (through the use of standard
libraries, for instance). They additionally provide tools that makes creating databases and handling
user management more efficient, both of which are key challenges the project faces. Django in
particular is well-suited for these needs, as it uses a Python back-end that is relatively simple to
learn and automatically solves many of the various security considerations. While there will be some
initial difficulty merging the front-end that we create using React with the back-end and server
under the framework, this initial overhead is acceptable considering the gains that will be made
further along in the development process.

\subsection{MySQL}
For the purposes of this project, any relation database management system available today would
present a suitable choice. However, given the wide availability of resources, its accessibility on
the host-server, and the server-development team's past experience with the software, MySQL will be
used to create the program's database, and manage user-data (which includes their login credentials,
contacts, and conversation histories).

\section{Configuration Management Plan}
\subsection{GitHub}
Given the scale of the project and the size of the group, a mechanism for sharing code is required
to ensure that development progresses efficiently. GitHub is an obvious choice for such a platform,
and will act as the main hub for the project. Other attendant benefits include the ability to track
any changes made to the code, revert to an earlier version in case of serious error, and ensure that
all progress is saved in the event of data loss. Lastly, the ubiquity of the platform and the number
of resources available that document its functionality means the overhead cost of learning to use
GitHub is relatively small compared to the benefits gained.

\subsection{Long-Term Management}
The long-term management of the project will fall to the system administrator, whose duties will
include user-management (including adding new users, and helping existing users with issues pertaining
to their accounts), adding or modifying the functionality and appearance of the program, and
resolving any issues. Along with editing the source files, there may be additional need to migrate
the project to a new host-machine in the future. The details of this is clarified in section 6
below.

\section{Risk Management Plan}
While it is difficult to assess potential risks while the details of the group's organisation and
ultimate goal are still being assessed, it is possible to identify several general obstacles that
the project might face:

\subsection{Withdrawal}
\textit{Probability:} Low (\textless20\%) \begin{addmargin}[1em]{2em} Given the nature of the course
(third year, no examinations, single large-scale project), and that the term is nearing its
completion are a third of the way, it is unlikely that a group member will withdraw from the course
going forward.\end{addmargin}\vspace{2mm}
\noindent \textit{Impact:} Low (2/5) \begin{addmargin}[1em]{2em}Though a withdrawal would
typically have a large negative impact on the progress of the group’s development, a large group
means that a single member’s responsibilities can be distributed across the team without putting too
much burden on the other members.\end{addmargin}\vspace{2mm}
\noindent \textit{Mitigation:} \begin{addmargin}[1em]{2em} Discuss with the team prior to delegating
responsibilities and beginning development to assess any potential challenges they might face both
in the course and outside of it that might push them to consider withdrawing, and
discuss possible solutions (assigning less work, getting extra support, etc.).\end{addmargin}

\subsection{Data Loss}
\textit{Probability:} Medium (50\%) \begin{addmargin}[1em]{2em} The loss of data, whether through
broken, lost, or faulty equipment, is a common occurrence, and one that is difficult to predict.
\end{addmargin}\vspace{2mm}
\noindent \textit{Impact:} High (4/5) \begin{addmargin}[1em]{2em} Without taking the proper
precautions, data loss presents one of the most substantial risks to the project, threatening
weeks of progress, and resulting in large set backs, or an inability to meet the requirements.
\end{addmargin}\vspace{2mm}
\noindent \textit{Mitigation:} \begin{addmargin}[1em]{2em} Protecting against data loss is becoming
increasingly easier with the wide availability of cloud services and the reduced cost of storage
media. Additionally, distributing new code to each member of the group using GitHub lessens any
potential impact.\end{addmargin}

\subsection{Late-Stage Changes in Client's Needs}
\textit{Probability:} High (\textgreater60\%) \begin{addmargin}[1em]{2em} A client’s needs change on
a regular basis, and this is particularly true for a product that is custom-designed for them. While
such a change is not itself a problem, a large-scale change late in the development cycle can pose a
significant setback.\end{addmargin}\vspace{2mm}
\noindent \textit{Impact:} Medium (3/5) \begin{addmargin}[1em]{2em} Depending on the
scale, and the time during the development cycle in which they are brought up,
this could represent a serious setback to the project, potentially requiring ground-up changes in
functionality or design. Alternatively, small-scale changes can be implemented at virtually any
point in the development cycle at little extra cost.\end{addmargin}\vspace{2mm}
\noindent \textit{Mitigation:} \begin{addmargin}[1em]{2em} Involve clients at every point of the
development cycle (within reason), determine a set of core needs that won’t change, and work on
addressing these first. Communicate with clients what a reasonable set of expectations for the
product looks like, and what the team is able to achieve given the time constraints. \end{addmargin}

\subsection{Failure to Meet Deadline}
\textit{Probability:} Very Low (\textless10\%) \begin{addmargin}[1em]{2em} With the size of the
group, the importance of the project, and the schedule created, it is unlikely that the base project
will be incomplete by the deadline.\end{addmargin}\vspace{2mm}
\noindent \textit{Impact:} Very High (5/5) \begin{addmargin}[1em]{2em} A non-functioning product, or
one that doesn't meet the client's specifications represents a significant failure to complete the
project. \end{addmargin}\vspace{2mm}
\noindent \textit{Mitigation:} \begin{addmargin}[1em]{2em} Create a detailed plan that specifies the
key features of the project. Ensure each group member has a clear idea of what their
responsibilities are, and where in the development process they should be at various stages.
Ensure resources are allocated appropriately, and address any problems as they
arise.\end{addmargin}

\section{Installation and Maintenance Plan}

The project is a web application that is be hosted on the undergraduate computer science student's
server at Saint Mary's University (\url{ugdev.cs.smu.ca/~group4}). The details of how this project's
databases and server-side components is implemented is clarified within the Installation and
Maintenance guide.

\pagebreak
\section*{Appendix}

\begin{figure}[H]
\includegraphics[scale=0.48]{gantt1.png}
\centering
\end{figure}

\begin{figure}[H]
\includegraphics[scale=0.48]{gantt2.png}
\centering
\end{figure}

\begin{figure}[H]
\includegraphics[scale=0.5]{gantt3.png}
\centering
\end{figure}

\end{document}
