\documentclass[11pt]{article}

\usepackage{color}
\usepackage{xcolor}
\usepackage{enumitem}
\usepackage{scrextend}
\usepackage{fancyhdr}
\usepackage{blindtext}
\usepackage{graphicx}
\usepackage{placeins}
\usepackage[utf8]{inputenc}
\usepackage[obeyspaces]{url}
\usepackage[margin=1.25in]{geometry}

\pagestyle{fancy}
\lfoot{\small\scshape SRS}
\cfoot{}
\rfoot{\footnotesize \thepage}
\lhead{\small\scshape CSCI 3428}
\chead{}
\rhead{\footnotesize Software Engineering}
\renewcommand{\headrulewidth}{.3pt}
\renewcommand{\footrulewidth}{.3pt}
\setlength\voffset{-0.25in}
\setlength\textheight{648pt}

\begin{document}

\title{CSCI 3428 - Software Requirements Specification}
\author{Group 4}
\date{Wednesday 11\textsuperscript{th} December, 2019}
\maketitle

\fancypagestyle{plain}{
\fancyhf{} % clear all header and footer fields
\fancyfoot[r]{\footnotesize \thepage} % except the right
\fancyfoot[l]{\small\scshape SRS} % and left
\renewcommand{\headrulewidth}{0pt}
\renewcommand{\footrulewidth}{.3pt}}

\section{Introduction}
\subsection{Purpose}
The program is intended to allow users to communicate with one another via text and images through
instant messaging. It distinguishes itself from other messaging platforms by prioritising
accessibility (by being tailored to the individual needs of each of the users), as well as
ease-of-use and simplicity. It hopes to respond to the need for simple and accessible web-based
services for use by the elderly.

\subsection{Intended Audience}
The program is being custom-designed for three residents of the Northwood Long-Term Care
facility in Halifax, Nova Scotia. While the program's functionality is similar in nature to any
other messaging platform, and can therefore be exploited by a wider user-group, its design will be
constrained according to the needs of the three residents, and will be driven based on the feedback
we receive from the residents during the testing and prototype phase.

\subsection{Intended Use}
The program is intended to be used as a text- and image- based communication platform. While it will not
include functionality for voice or video communication between users, it might implement
accessibility features that allow the users to control and interact with the program by voice,
depending on their specific needs.

\section{Description}
The system is a free instant-messaging platform tailored to the specific needs of residents at the
Northwood long-term care facility.

\subsection{User Needs}
Users of the platform require an instant messaging system that allows them to communicate with
friends and family via the internet. Specific demands include the ability to send and receive text
messages and images to and from a number of different contacts, using separate conversations. In
prioritising simplicity, the system will have no outward-facing functionality that will allow users
to add or delete contacts. This functionality, among other considerations, are left to the system
administrator's control, via the user-management panel.

\section{System Features and Requirements}

\begin{figure}[!htb]
  \centering
  \includegraphics[scale=0.4]{component.png}
  \caption{System Component Diagram}
\end{figure}

\subsection{External Interface Requirements}
\subsubsection{User Interfaces}
There are two primary user-interfaces that the users will interact with. The first is a log-in
screen, for security and user-differentiation purposes, which gives each user access to their own
conversation list. The user accesses the system via a given username (their first name) and
password. The second primary user-interface is the conversation panel, which lists all currently active
conversations that user has. The selected conversation appears to the right of the panel, and allows
the user to scroll through their entire conversation history, as well as toggle between viewing the
entire conversation (messages and images), and only the images they have sent or received. This
toggle is activated by clicking the image icon that appears in the top-right corner of the chat
window. The conversation panel include a text-box and image-upload button that allows them to send
text and images respectively to their selected recipient.\newline

\begin{figure}[!htb]
  \centering
  \includegraphics{login}
  \caption{Log-In Page}
\end{figure}

\noindent \textbf{Details for Login Page:}
\begin{enumerate}
    \item Username (name): The user is required to log in with their username credential in order to
        access the system. Text, size: 30
    \item Password (password): The user is required to log in with their password credential in
        order to access the system. Text, size: 50
    \item The correct combination of a username and its corresponding password is required to be
        entered into the system.
\end{enumerate}

\begin{figure}
    \centering
    \includegraphics{chat}
    \caption{Conversation Panel}
\end{figure}
\pagebreak

\noindent \textbf{Details for Messaging Page:}
\begin{enumerate}
    \item Content of the Text-box (content): The user may type text into the input box on the bottom
        right of the screen. Upon pressing `Enter', the text is displayed in the conversation
        window. The text is then sent to the recipient (another user of the system) who can view
        received messages on the left side of the screen in the chronological order of which they
        were received. Text, size: 100
    \item Image Upload button: To the left of the text-box is an image-upload button, that allows
        them to send images to within their current conversation. Upon clicking, a their operating
        system's file browser will display, prompting them to select their desired image. Once
        selected, the image is automatically uploaded to the server, and sent to the recipient.
        Image file, size: N/A
\end{enumerate}
\FloatBarrier

\subsubsection{Software Interfaces}
The product will be accessed via a web-browser, and can therefore be widely used on most modern
hardware. Specific requirements include in-browser support for the latest standards of both HTML
(HTML5) and CSS (CSS3), as well as JavaScript and/or Python web-scripts. Ideal choices include the
latest desktop versions of both Chrome (\url{v.78}) and Firefox (\url{v.70}), as the project was
created and tested on both of these platforms.

\subsubsection{Communication Interfaces}
The project's webpage will be hosted on the undergraduate student's server at Saint Mary's
University, and is accessible at \url{ugdev.cs.smu.ca/~group4}. Upon completion of the project, the
system administrator will be given access to the server, its attendant SQLite database, and the
user-management panel, to allow for long-term maintenance of the project. See the `Installation
and Maintenance' document for further details.

\subsection{Functional Requirements}

\begin{figure}[!htb]
  \centering
  \includegraphics[scale=0.75]{tree}
  \caption{Use-case diagram}
\end{figure}

\subsubsection{Essential}
The following list represents core-aspects of the system's functionality that must be present in
order to satisfy the client's requirements for its use.
\begin{enumerate}
    \item Authenticate and log-in user into system: secures each user account to prevent public
        access to their communications. A robust user-management framework will further allow
        wider use of the project outside of its initial scope.
    \item Allow user to choose and change who they communicate with: provide functionality to allow
        users to open (or close existing) channels of communication with other users on the
        platform. Provides greater flexibility, and a potentially wider use-case for the project.
    \item Allow user to send text messages to others: the key use-requirement for the entire system.
    \item Allow user to send images to others: the second key use-requirement for the entire system.
    \item Allow user to receive text messages from others: an extension of the first key
        use-requirement.
    \item Allow user to receive images from others: an extension of the second key use-requirement.
    \item Display error message if connection to server fails: failure to deliver potentially
        important messages can have serious consequences for the user, so it is crucial to notify
        the user should the system fail to function, and ideally provide some indication for how the
        issue can be resolved (i.e.  differentiate between no internet connection vs. a problem on
        the server's end).
    \item Enable user to log-out of system: allows use on public machines (e.g. at a library)
        without compromising the user's private data and conversations.
\end{enumerate}

\subsection{Performance and Quality Requirements}
\subsubsection{Essential}
The following represents a list of requirements that must be present in order to distinguish it from
alternative options. While not explicitly-stated by the core user's as a requirement, the following
are necessary features to ensure the system is usable and maintainable over the duration of its
life-span (minimum: 1-year).
\begin{enumerate}
    \item Display all messages sent by the user on the right of the conversation panel, and
        the messages sent by their participant on the left.
    \item Display time-stamp of when message was sent.
    \item Display conversation history in order that messages were sent (oldest appears at the top,
        newest appears at the bottom).
    \item Perform real-time synchronisation between user's session and database to update the
        conversation panel and display new messages as they are sent, without requiring user to
        refresh the page.
    \item Allow the system-administrator to perform various user-management tasks:
        \subitem i. Change user's password
        \subitem ii. Add a new conversation between two users
        \subitem iii. Delete an existing conversation between two users
        \subitem iv. Add new user
        \subitem v. Delete existing user
\end{enumerate}

\subsubsection{Optional}
The following list represents a set of requirements that would make the system more flexible,
allowing for use that can be tailored to a wider audience. Due to the time constraints present
during this project, this functionality was not implemented in the first release of the system to
its primary users, but remains an option for expansion for later in its life-span.
\begin{enumerate}
    \item Allow user to access a settings page: provide functionality to allow users to tailor the
        appearance of the product according to their changing needs. Enables new users to access and
        use the system according to their own needs.
    \item Allow user to change font size for messaging interface: improve readability of the
        displayed text according to user's preferences.
    \item Allow user to change background colour of messaging interface: improve readability,
        comfort, and appearance of the system according to user's preferences.
    \item Allow user to change text colour of messaging interface: improve readability and
        appearance of the displayed text according to user's preferences.
\end{enumerate}

\subsection{Design Constraints}
\subsubsection{Essential}
The following is a general list of constraints that are placed on design of the system:
\begin{enumerate}
    \item System must be browser-based, and be able to run on modern browsers (i.e. Google Chrome or
        Mozilla Firefox).
    \item System must be hosted on the SMU computer science undergraduate student's server at
        \url{ugdev.cs.smu.ca/~group4}.
    \item System must validate against current standards for web-development (specifically HTML5 and
        CSS3)
\end{enumerate}

The following list represents the set of user-specific constraints that are placed on the design and
functionality of the system, which must be taken into consideration in order to deliver a final
product that is in line with the core user's needs.
\begin{enumerate}[label=\alph*.]
    \item \textbf{Valerie}
        \subitem i. System must be readable and usable on a tablet screen.
        \subitem ii. Must use varied colour palette to distinguish elements for readability.
        \subitem iii. System must have support for tablet-based browsing platforms.
    \item \textbf{May}
        \subitem i. Must allow user to easily navigate through message history.
        \subitem ii. Must allow user to easily navigate through image history.
    \item \textbf{Bob}
        \subitem i. System must include a high contrast between background and text.
        \subitem ii. Background must be light-grey, and foreground elements/text must be lighter.
        \subitem iii. Must allow users to have separate conversations with different users (family).
\end{enumerate}

\end{document}
