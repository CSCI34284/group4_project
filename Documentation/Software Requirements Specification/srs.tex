\documentclass[11pt]{article}

\usepackage{color}
\usepackage{xcolor}
\usepackage{enumitem}
\usepackage{scrextend}
\usepackage{fancyhdr}
\usepackage{blindtext}
\usepackage{graphicx}
\usepackage{placeins}
\usepackage[utf8]{inputenc}
\usepackage[obeyspaces]{url}
\usepackage[margin=1.25in]{geometry}

\pagestyle{fancy}
\lfoot{\small\scshape SDP}
\cfoot{}
\rfoot{\footnotesize \thepage}
\lhead{\small\scshape CSCI 3428}
\chead{}
\rhead{\footnotesize Software Engineering}
\renewcommand{\headrulewidth}{.3pt}
\renewcommand{\footrulewidth}{.3pt}
\setlength\voffset{-0.25in}
\setlength\textheight{648pt}

\begin{document}

\title{CSCI 3428 - Software Requirements Specification}
\author{Group 4}
\date{Monday 18\textsuperscript{rd} November 2019}
\maketitle

\fancypagestyle{plain}{
\fancyhf{} % clear all header and footer fields
\fancyfoot[r]{\footnotesize \thepage} % except the right
\fancyfoot[l]{\small\scshape SRS} % and left
\renewcommand{\headrulewidth}{0pt}
\renewcommand{\footrulewidth}{.3pt}}

\section{Introduction}
\subsection{Purpose}
The program is intended to allow users to communicate with each other via text and images through
instant messaging. It distinguishes itself from other messaging platforms by prioritising
accessibility (by being tailored to the individual needs of each of the users), as well as
ease-of-use and simplicity. It hopes to respond to the need for simple and accessible web-based
services for use by the elderly.

\subsection{Intended Audience}
The program is being custom-designed for three residents of the Northwood Long-Term Care
facility in Halifax, Nova Scotia. While the program's functionality is similar in nature to any
other messaging platform, and can therefore be exploited by a wider user-group, its design will be
constrained according to the needs of the three residents, and will be driven based on the feedback
we receive from the residents during the testing and prototype phase.

\subsection{Intended Use}
The program is intended to be used as a text- and image- based communication platform. It will not
include functionality for voice or video communication between users, however it might implement
accessibility features that allow the users to interact with the program by voice, depending on
their specific needs.

\section{Description}
The system is a free instant-messaging platform tailored to the specific needs of residents at the
Northwood long-term care facility.

\subsection{User Needs}
Users of the platform require an instant messaging system that allows them to communicate with
friends and family via the internet. Specific demands include the ability to send and receive text
messages and images from a number of different contacts, using separate conversations. Users should
be able to add or remove contacts as they see fit, and open or close lines of communication with
these users.
\section{System Features and Requirements}
\subsection{External Interface Requirements}
\subsubsection{User Interfaces}
There are two primary user-interfaces that the users will interact with. The first is a log-in
screen, which allows us to distinguish between users, and gives each user access to their own
conversation list. Depending on the user's needs, it may not be required to use a password to
authenticate the log-in. The second is the conversation panel, which lists on the left all active
conversations that user has. The currently selected conversation appears on the right, and allows
the user to scroll through their entire conversation history, as well as toggle between viewing the
entire conversation, and only the images they have sent or received. This toggle is activated by
clicking the image icon that appears in the top-right corner of the chat window.

\begin{figure}[!htb]
  \includegraphics{login}
  \caption{Log-In Page}
\end{figure}

\noindent \textbf{Details for Login Page:}
\begin{enumerate}
    \item Username (name): The user is required to log in with their username credential in order to
        access the system. Text, size: 30
    \item Password (password): The user is required to log in with their password credential in
        order to access the system. Text, size: 50
    \item A correct combination of a username and its corresponding password is required to be
        entered to access the system.
\end{enumerate}

\begin{figure}
    \includegraphics{chat}
    \caption{Conversation Panel}
\end{figure}
\pagebreak

\noindent \textbf{Details for Messaging Page:}
\begin{enumerate}
    \item Content of the Textbox (content): The user may type text to the input box on the bottom
        right of the screen. Upon pressing enter, the text is displayed on the right side of the
        screen. The text is then sent to the recipient, another user of the system, who can view
        received messages on the left side of the screen in the chronological order of which they
        were received. Text, size: 100
\end{enumerate}
\FloatBarrier

\subsubsection{Software Interfaces}
The product will be accessed via a web-browser, and can therefore be widely used on most modern
hardware. Specific requirements include in-browser support for the latest standards of both HTML
(HTML5) and CSS (CSS3), as well as JavaScript and/or Python web-scripts. Ideal choices include the
latest desktop versions of both Chrome (\url{v.78}) and Firefox (\url{v.70}), as the project was
both created and tested on these platform.

\subsubsection{Communication Interfaces}
The project's webpage will be hosted on the undergraduate student's server at Saint Mary's
University, and is accessible at \url{ugdev.cs.smu.ca}. Upon completion of the project, the system
administrator will be given access to the server, its attendant MySQL database, and the
user-management framework, to allow for long-term maintenance of the project. See the `Installation
and Maintenance' document for further details.

\subsection{Functional Requirements}

\begin{figure}[!htb]
  \includegraphics{tree}
  \caption{Use-case diagram}
\end{figure}

\subsubsection{Essential}
The following list represents core-aspects of the system's functionality that must be present in
order to satisfy the client's requirements, and allow the system to be usable and maintainable over
the project's life-span (at minimum: 1-year).
\begin{enumerate}
    \item Authenticate and log-in user into system: secures each user account to prevent public
        access to their communications. A robust user-management framework will further allow
        wider use of the project outside of its initial scope.
    \item Allow user to choose and change who they communicate with: provide functionality to allow
        users to open (or close existing) channels of communication with other users on the
        platform. Provides greater flexibility, and a potentially wider use-case for the project.
    \item Allow user to send text messages to others: the key use-requirement for the entire system.
    \item Allow user to send images to others: the second key use-requirement for the entire system.
    \item Allow user to receive text messages from others: an extension of the first key
        use-requirement.
    \item Allow user to receive images from others: an extension of the second key use-requirement.
    \item Display error message if connection to server fails: failure to deliver potentially
        important messages can have serious consequences for the user, so it is crucial to notify
        the user should the system fail to function, and ideally provide some indication for how the
        issue can be resolved (i.e.  differentiate between no internet connection vs. a problem on
        the server's end).
    \item Enable user to log-out of system: allows use on public machines (e.g. at a library)
        without compromising the user's private data and conversations.
\end{enumerate}

\subsubsection{Desirable}
The following list represents a set of requirements that would make the system more flexible,
allowing for use that can be tailored to a wider audience. While not an explicit requirements of the
current three clients, such functionality could expand the project's use.
\begin{enumerate}
    \item Allow user to access a settings page: provide functionality to allow users to tailor the
        appearance of the product according to their changing needs. Enables new users to access and
        use the system according to their own needs.
    \item Allow user to change font size for messaging interface: improve readability of the
        displayed text according to user's preferences.
    \item Allow user to change background colour of messaging interface: improve readability,
        comfort, and appearance of the system according to user's preferences.
    \item Allow user to change text colour of messaging interface: improve readability and
        appearance of the displayed text according to user's preferences.
\end{enumerate}

\subsection{Performance Requirements}
The messaging system will be browser-based, and run from the \url{ugdev.cs.smu.ca} server. Initial
load time will be dependent on the internet connection available to the user, and the stability of
the hosting server. The performance will additionally be marginally dependent on the hardware
available to the user to interact with the program. Additionally, there exist the following
performance requirements (according to the needs of the system's three core-users):
\begin{enumerate}[label=\alph*.]
    \item \textbf{Valerie}
        \subitem System must be readable and usable on a small screen
        \subitem Use of varied colour palette to distinguish elements for readability
        \subitem Support for mobile browsing platforms
    \item \textbf{May}
        \subitem Ability to favourite/save important messages
        \subitem Ability to easily navigate through message history/browse old messages
        \subitem Ability to easily navigate through image history/browse old images
    \item \textbf{Bob}
        \subitem High contrast between background and text
        \subitem Background must be dark, and foreground elements/text must be light
        \subitem Ability to have separate conversations with different family members
\end{enumerate}

\subsection{Design Constraints}
Users must have access to a modern web browser (e.g. Google Chrome, Mozilla Firefox, Safari, or
Internet Explorer) that is able to support the program. This includes being compatible with HTML5,
CSS3, and being able to run JavaScript- or Python-based scripts. Additionally, the user will need an
active internet connection to be able to make use of the program. From the development perspective,
the program is constrained according to compatibility with the \url{ugdev.cs.smu.ca} server. That
includes using databases that are available to be installed on the server.

\end{document}
